\begin{exercise}
Let \(I\) be a set.
A \textbf{partition} of \(I\) is a collection \(Q \subseteq P(I)\) of subsets of \(I\) satisfying (1) every set in \(Q\) is non-empty (2) if \(a, b \in Q\) and \(a \cap b \neq \emptyset\), then \(a = b\) (3) \(\bigcup Q = I\).
A \textbf{partial partition} of \(I\) is a subset \(Q \subseteq P(I)\) satisfying (1) and (2) but not necessarily (3).
Show that
\begin{parts}
\part
there is a bijective correspondence between partitions and equivalence
relations and between partial partitions and partial equivalence relations (on \(I\));
\part
there is a bijective correspondence between partial equivalence relations on \(I\) and equivalence relations on subsets of \(I\).
\end{parts}
\end{exercise}

\begin{partsolution}{i}
For a set \(I\), let \(\mathrm{PPart}(I)\) and \(\mathrm{PER}(I)\) denote the sets of partial partitions on \(I\) and partial equivalence relations on \(I\), respectively.
Within these, let \(\mathrm{Part}(I) \subseteq \mathrm{PPart}(I)\) and \(\mathrm{ER}(I) \subseteq \mathrm{PER}(I)\) denote the sets of (total) partitions on \(I\) and (total) equivalence relations on \(I\), respectively.
Define
\begin{align*}
f &: \mathrm{PPart}(I) \to \mathrm{PER}(I), &
g &: \mathrm{PER}(I) \to \mathrm{PPart}(I)
\end{align*}
by
\begin{align*}
f(Q) &= \big\{(x, y) \in I^2 : \text{there exists \(a \in Q\) such that \(x, y \in a\)}\big\}, \\
g(R) &= \big\{\text{equivalence classes of \(R\)}\big\}.
\end{align*}
A routine verification shows that \(f\) and \(g\) are mutual inverses, and furthermore they restrict to mutually inverse bijections between \(\mathrm{Part}(I)\) and \(\mathrm{ER}(I)\).
\end{partsolution}

\begin{partsolution}{ii}
Note that for distinct subsets \(J\) and \(K\) of \(I\), the sets \(\mathrm{ER}(J)\) and \(\mathrm{ER}(K)\) are disjoint, so we have
\begin{equation*}
\mathrm{PER}(I)
= \bigcup_{J \subseteq I} \mathrm{ER}(J)
\cong \coprod_{J \subseteq I} \mathrm{ER}(J).
\qedhere
\end{equation*}
\end{partsolution}