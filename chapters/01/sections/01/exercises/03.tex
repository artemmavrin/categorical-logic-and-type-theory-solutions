\begin{exercise}
Consider the total category of a fibration.
Show that
\begin{parts}
\part
every morphism factors as a vertical map followed (diagrammatically) by a Cartesian one;
\part
a Cartesian map above an isomorphism is an isomorphism.
Especially a vertical Cartesian map is an isomorphism.
\end{parts}
\end{exercise}

\begin{partsolution}{i}
Let \(\cat{E}\) be a fibration over \(\cat{B}\), and let \(f : X \to Y\) be a morphism in \(\cat{E}\) over a morphism \(u : I \to J\) in \(\cat{B}\).
Then there exists a Cartesian morphism \(f^\prime : X^\prime \to Y\) in \(\cat{E}\) above \(u\).
Then we are in the situation of the diagram on the left below, so there exists a unique (vertical) morphism \(g : X \to X^\prime\) over \(I\) such that the diagram below on the right commutes.
\begin{equation*}
\begin{tikzcd}[row sep=small]
X \arrow[rrd, bend left, "f"] \arrow[dd, -Triangle] \\
& X^\prime \arrow[r, "f^\prime"] & Y \arrow[dd, -Triangle] \\
I \arrow[rd, equal] \arrow[rrd, bend left, near end, "u"] \\
& I \arrow[r, "u"] & J
\arrow[from=uul, to=l, -Triangle, crossing over]
\end{tikzcd}
\implies
\begin{tikzcd}[row sep=small]
X \arrow[rrd, bend left, "f"] \arrow[dd, -Triangle] \arrow[rd, dashed, "g"] \\
& X^\prime \arrow[r, "f^\prime"] & Y \arrow[dd, -Triangle] \\
I \arrow[rd, equal] \arrow[rrd, bend left, near end, "u"] \\
& I \arrow[r, "u"] & J
\arrow[from=uul, to=l, -Triangle, crossing over]
\end{tikzcd}
\end{equation*}
In particular, \(f = f^\prime \circ g\); in other words, \(f\) factors as a vertical morphism followed by a Cartesian morphism.
\end{partsolution}

\begin{partsolution}{ii}
Let \(\cat{E}\) be a fibration over \(\cat{B}\), and let \(f : X \to Y\) be a Cartesian morphism in \(\cat{E}\) over an isomorphism \(u : I \to J\) in \(\cat{B}\).
Let \(v = u^{-1} : J \to I\) in \(\cat{B}\), and consider the situation in the following commutative diagram.
\begin{equation*}
\begin{tikzcd}[row sep=small]
Y \arrow[rd, dashed, "\exists! g"] \arrow[rrd, bend left, "\id_Y"] \arrow[dd, -Triangle] \\
& X \arrow[r, "f"] & Y \arrow[dd, -Triangle] \\
J \arrow[rd, "v"{below left}, "\sim"{sloped}] \arrow[rrd, bend left, near end, "\id_J"] \\
& I \arrow[r, "u"{below}, "\sim"] & J
\arrow[from=uul, to=l, -Triangle, crossing over]
\end{tikzcd}
\end{equation*}
Here the dashed morphism \(g: Y \to X\) exists uniquely by virtue of \(f\) being Cartesian.
Since \(f \circ g = \id_Y\) is Cartesian (by \expartref{1}{1}{4}{i}) and \(f\) is Cartesian, \expartref{1}{1}{4}{iii} implies that \(g\) is Cartesian over the isomorphism \(v\), and \(f \circ g = \id_Y\).
The same argument then implies that there exists a morphism \(h : X \to Y\) over \(u = v^{-1}\) such that \(g \circ h = \id_X\).
But then
\begin{equation*}
f = f \circ \id_X = f \circ g \circ h = \id_Y \circ h = h,
\end{equation*}
so \(g \circ f = \id_X\), and hence \(g = f^{-1}\), so \(f\) is an isomorphism.
\end{partsolution}