\begin{exercise}
For an arbitrary category \(\cat{B}\), consider the domain functor \(\dom : \cat{B}^{\rightarrow} \to \cat{B}\).
\begin{parts}
\part
Describe the fibre category above \(I \in \cat{B}\).
It is usually called the \textbf{opslice category} or simply \textbf{opslice} and written as \(I \setminus \cat{B}\).
\part
Show that \(\dom\) is a fibration (without any assumptions about \(\cat{B}\)).
\part
Show also that for each \(I \in \cat{B}\) the domain functor \(\dom_I : \cat{B} / I \to \cat{B}\) is a fibration.
\end{parts}
\end{exercise}

\begin{partsolution}{i}
For a category \(\cat{B}\) and an object \(I \in \cat{B}\), the category \(I \setminus \cat{B}\) has
\begin{description}
\item[objects]
Morphisms \(\begin{pmatrix}\begin{tikzcd}[row sep=small] I \arrow[d, "\varphi"] \\ X\end{tikzcd}\end{pmatrix}\) in \(\cat{B}\) with domain \(I\).
\item[morphisms]
Given objects \(\begin{pmatrix}\begin{tikzcd}[row sep=small] I \arrow[d, "\varphi"] \\ X\end{tikzcd}\end{pmatrix}\) and \(\begin{pmatrix}\begin{tikzcd}[row sep=small] I \arrow[d, "\psi"] \\ Y\end{tikzcd}\end{pmatrix}\) in \(I \setminus \cat{B}\), a morphism in \(I \setminus \cat{B}\) from the first to the second consists of a morphism \(f : X \to Y\) in \(\cat{B}\) such that the following diagram commutes.
\begin{equation*}
\begin{tikzcd}[column sep=small]
& I \arrow[dl, "\varphi"{above left}] \arrow[dr, "\psi"] \\
X \arrow[rr, "f"] && Y
\end{tikzcd}
\qedhere
\end{equation*}
\end{description}
\end{partsolution}