\begin{exercise}
Check that the following are (trivial) examples of fibrations
\begin{equation*}
\begin{tikzcd}[%
  , row sep=scriptsize
  , column sep=large
  ]
\cat{B} \times \cat{C} \arrow[d, "\fst"]
& \cat{B} \arrow[d, "\id"]
& \cat{B} \arrow[d]
&[-4.2em]
&[-2em] X \arrow[d]
\\
\cat{B}
& \cat{B}
& 1
& = \{*\}
& I
\end{tikzcd}
\end{equation*}
where \(X\), \(I\) are sets (\emph{i.e.} discrete categories).
\end{exercise}

\begin{solution}
Consider the functor \(\cat{B} \times \cat{C} \xrightarrow{\fst} \cat{B}\), which projects onto the first component.
Consider a morphism \(u : I \to J\) in \(\cat{B}\) and an object \((J, Y)\) in \(\cat{B} \times \cat{C}\) above \(J\).
We claim that \((v, \id_Y) : (I, Y) \to (J, Y)\) is a Cartesian lift of \(u\).
Suppose there are morphisms \(v : K \to J\) in \(\cat{B}\) and \((w, g) : (K, Z) \to (J, Y)\) in \(\cat{B} \times \cat{C}\) above \(u \circ v\) (i,e, \(w = u \circ v\)), as in the following diagram.
\begin{equation*}
\begin{tikzcd}[row sep=small]
(K, Z) \arrow[rd, dashed, "?"] \arrow[rrd, bend left, "({w, g)}"] \arrow[dd, -Triangle, "\fst"] \\
& (I, Y) \arrow[r, "{(u, \id_Y)}"] & (J, Y) \\
K \arrow[rd, "v"] \\
& I \arrow[r, "u"] & J
\arrow[from=uul, to=l, -Triangle, crossing over, "\fst"]
\arrow[from=uu, -Triangle, crossing over, "\fst"]
\end{tikzcd}
\end{equation*}
Observe that \((v, g) : (K, Z) \to (I, Y)\) is the unique morphism in \(\cat{B} \times \cat{C}\) that fills in the dashed ``\(?\)'' morphism making the diagram above commute, so \((u, \id_Y)\) is Cartesian and hence \(\cat{B} \times \cat{C} \xrightarrow{\fst} \cat{B}\) is a fibration.

Next, consider the identity functor \(\cat{B} \xrightarrow{\id} \cat{B}\) and a morphism \(u : I \to J\) in \(\cat{B}\).
The only object above \(J\) in this scenario is \(J\) itself, and we claim that \(u\) itself is the unique Carteisan lift of \(u\).
Indeed, given morphisms \(v : K \to I\) in \(\cat{B}\) (``below'') and \(w : K \to J\) in \(\cat{B}\) (``above'') over \(u \circ v\) (i.e., \(w = u \circ v\)), the unique morphism \(K \to I\) which fills in the dashed ``?'' arrow in the diagram below is \(v\) itself.
\begin{equation*}
\begin{tikzcd}[row sep=small]
K \arrow[rd, bend left, dashed, near end, "?"] \arrow[rd, bend right, "v"{below}] \arrow[rrd, bend left, "w"] \\
& I \arrow[r, "u"] & J
\end{tikzcd}
\end{equation*}
Thus, \(u\) is a Cartesian lift of \(u\), and \(\cat{B} \xrightarrow{\id} \cat{B}\) is a fibration.

Consider the functor \(\cat{B} \to 1\), and observe that this situation is equivalent to the functor \(1 \times B \xrightarrow{\fst} 1\), so it is a fibration by the first part of this exercise.
More explicitly, consider \emph{the} morphism \(* \to *\) in \(1\), and let \(Y\) be an object in \(\cat{B}\) (vacuously above \(*\)).
Then \(\id_Y : Y \to Y\) is a Cartesian lift of \(* \to *\), whence \(\cat{B} \to 1\) is a fibration.

Finally, let \(X\) and \(Y\) be sets and \(p : X \to I\) a function, considered as a functor between discrete categories.
All morphisms in \(I\) and \(X\) are identities, so if \(i \in I\) and \(x \in X\) is a ``lift'' of \(i\) (i.e., \(p(x) = i\) or equivalently \(x \in p^{-1}(i)\)), then \(\id_x\) is a Cartesian lift of \(\id_i\).
Therefore, \(p : X \to I\) is a fibration.
\end{solution}