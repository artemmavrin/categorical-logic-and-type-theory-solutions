\begin{exercise}~ %
\begin{parts}
\part
Prove Proposition~1.1.4.\footnote{%
\cite[Proposition~1.1.4]{MR1674451} asserts that Cartesian liftings are unique up-to-isomorphism (in a slice): if \(f\) and \(f^\prime\) with \(\cod f = \cod f^\prime\) are both Cartesian over the same map, then there is a unique vertical isomorphism \(\varphi : X \xrightarrow{\cong} X^\prime\) with \(f^\prime \circ \varphi = f\).%
}
\part
Suppose \(f\) is Cartesian and \(g\) and \(h\) are above the same map.
Show that \(f \circ g = f \circ h\) implies \(g = h\).
\end{parts}
\end{exercise}

\begin{remark}
We will prove this in the more general context of displayed categories (cf.~\cite{lmcs:5252}).
The conclusion of \expartref{1}{1}{1}{i} (i.e., \cite[Proposition~1.1.4]{MR1674451}) appears in the literature in, e.g., \cite[Lemma~8.1.4(2)]{MR1313497}, \cite[Remark~3.3]{MR2222646}, and \cite[Lemma~2.3(ii)]{MR0213413}.
\end{remark}

\begin{partsolution}{i}
Suppose \(\cat{E}\) is a displayed category over \(\cat{B}\), \(u : I \to J\) is a morphism in \(\cat{B}\), and \(f : X \to Y\) and \(f^\prime : X^\prime \to Y\) are Cartesian morphisms in \(\cat{E}\) over \(u\).
Since \(f^\prime\) is Cartesian and \(f\) is a morphism into \(Y\) over \(u = u \circ \id_I\), there exists a unique morphism \(\varphi : X \to X^\prime\) in \(\cat{E}\) over \(I\) such that \(f = f^\prime \circ \varphi\).
This is summarized by the following diagram.
\begin{equation*}
\begin{tikzcd}[row sep=small]
X \arrow[rrd, bend left, "f"] \arrow[dd, -Triangle] \\
& X^\prime \arrow[r, "f^\prime"] & Y \arrow[dd, -Triangle] \\
I \arrow[rd, equal] \arrow[rrd, bend left, near end, "u"] \\
& I \arrow[r, "u"] & J
\arrow[from=uul, to=l, -Triangle, crossing over]
\end{tikzcd}
\implies
\begin{tikzcd}[row sep=small]
X \arrow[rrd, bend left, "f"] \arrow[dd, -Triangle] \arrow[rd, dashed, "\exists!\,\varphi"] \\
& X^\prime \arrow[r, "f^\prime"] \arrow[dd, -Triangle] & Y \arrow[dd, -Triangle] \\
I \arrow[rd, equal]\\
& I \arrow[r, "u"] & J
\end{tikzcd}
\end{equation*}
Similarly, there exists a unique morphism \(\psi : X^\prime \to X\) in \(\cat{E}\) over \(I\) such that \(f^\prime = f \circ \psi\).
Moreover, both \(\id_{X^\prime}\) and \(\varphi \circ \psi\) are morphisms in \(\cat{E}\) over \(I\) whose composition with \(f^\prime\) is \(f^\prime\) itself, as suggested in the following diagram.
\begin{equation*}
\begin{tikzcd}[row sep=small]
X^\prime \arrow[rd, "\psi"] \arrow[rrrdd, bend left, near start, "f^\prime"] \arrow[dd, -Triangle] \\
& X \arrow[rd, "\varphi"] \arrow[rrd, bend left, near start, "f"] \arrow[dd, -Triangle] \\
I \arrow[rd, equal] & & X^\prime \arrow[dd, -Triangle] \arrow[r, "f^\prime"] & Y \arrow[dd, -Triangle] \\
& I \arrow[rd, equal] \\
& & I \arrow[r, "u"] & J
\arrow[from=llluuuu, to=luu, bend right, crossing over, near start, "\id_{X^\prime}"{below}]
\end{tikzcd}
\end{equation*}
Thus, since \(f^\prime\) is Cartesian, \(\varphi \circ \psi = \id_{X^\prime}\).
By symmetry, \(\psi \circ \varphi = \id_X\), and hence \(\varphi\) is an isomorphism.
\end{partsolution}

\begin{partsolution}{ii}
Suppose \(\cat{E}\) is a displayed category over \(\cat{B}\), \(u : I \to J\) and \(v : K \to I\) are morphisms in \(\cat{B}\), \(f : X \to Y\) is a Cartesian morphism in \(\cat{E}\) over \(u\), and \(g, h : Z \to X\) are morphisms in \(\cat{E}\) over \(v\) such that \(f \circ g = f \circ h\).
These suppositions are summarized in the following commutative diagram.
\begin{equation*}
\begin{tikzcd}[row sep=small]
Z \arrow[rd, bend left, near end, "g"] \arrow[rd, bend right, "h"{below}] \arrow[dd, -Triangle] \arrow[rrd, bend left, near end, "f \circ g = f \circ h"] \\
& X \arrow[dd, -Triangle] \arrow[r, "f"] & Y \arrow[dd, -Triangle] \\
K \arrow[rd, "v"] \arrow[rrd, bend left, near end, "u \circ v"] \\
& I \arrow[r, "u"] & J
\arrow[from=uul, to=l, -Triangle, crossing over]
\end{tikzcd}
\end{equation*}
Since \(f\) is Cartesian, this situation immediately implies that \(g = h\).
\end{partsolution}