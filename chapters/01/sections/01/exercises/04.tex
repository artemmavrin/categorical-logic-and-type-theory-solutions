\begin{exercise}
Let
\begin{tikzcd}[row sep=small]
\cat{E} \arrow[d, "p"] \\ \cat{B}
\end{tikzcd}
be a fibration.
Prove that
\begin{parts}
\part
all isomorphisms in \(\cat{E}\) are Cartesian;
\part
if \(\xrightarrow{f}\xrightarrow{g}\) is a composable pair of Cartesian morphisms in \(\cat{E}\), then also their composite \(\xrightarrow{g \circ f}\) is Cartesian.
\end{parts}
Hence it makes sense to talk about the subcategory \(\Cart(\cat{E}) \hookrightarrow \cat{E}\) having all objects from \(\cat{E}\) but only the Cartesian arrows.
We write \(|p|\) for the composite \(\Cart(\cat{E}) \hookrightarrow \cat{E} \to \cat{B}\).
\begin{parts}[resume]
\part
Let \(\xrightarrow{f}\xrightarrow{g}\) be a composable pair in \(\cat{E}\) again.
Show now that if \(g\) and \(g \circ f\) are Cartesian, then \(f\) is Cartesian as well.
\part
Verify that a consequence of \partref{1}{1}{4}{iii} is that the functor \(|p| : \Cart(\cat{E}) \to \cat{B}\) is a fibration.
From \partref{1}{1}{3}{ii} in the previous exercise it follows that all fibres of \(|p|\) are groupoids (\emph{i.e.} that all maps in the fibres are isomorphisms).
\end{parts}
[This \(|p|\) will be called the \textbf{fibration of objects} of \(p\).]
\end{exercise}