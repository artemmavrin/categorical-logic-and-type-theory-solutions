\begin{exercise}
Let
\begin{tikzcd}[row sep=small]
\cat{E} \arrow[d, "p"] \\ \cat{B}
\end{tikzcd}
be a fibration.
Prove that
\begin{parts}
\part
all isomorphisms in \(\cat{E}\) are Cartesian;
\part
if \(\xrightarrow{f}\xrightarrow{g}\) is a composable pair of Cartesian morphisms in \(\cat{E}\), then also their composite \(\xrightarrow{g \circ f}\) is Cartesian.
\end{parts}
Hence it makes sense to talk about the subcategory \(\Cart(\cat{E}) \hookrightarrow \cat{E}\) having all objects from \(\cat{E}\) but only the Cartesian arrows.
We write \(|p|\) for the composite \(\Cart(\cat{E}) \hookrightarrow \cat{E} \to \cat{B}\).
\begin{parts}[resume]
\part
Let \(\xrightarrow{f}\xrightarrow{g}\) be a composable pair in \(\cat{E}\) again.
Show now that if \(g\) and \(g \circ f\) are Cartesian, then \(f\) is Cartesian as well.
\part
Verify that a consequence of \partref{1}{1}{4}{iii} is that the functor \(|p| : \Cart(\cat{E}) \to \cat{B}\) is a fibration.
From \partref{1}{1}{3}{ii} in the previous exercise it follows that all fibres of \(|p|\) are groupoids (\emph{i.e.} that all maps in the fibres are isomorphisms).
\end{parts}
[This \(|p|\) will be called the \textbf{fibration of objects} of \(p\).]
\end{exercise}

\begin{remark}
We will prove this in the more general context of displayed categories (cf.~\cite{lmcs:5252}).
\end{remark}

\begin{partsolution}{i}
Let \(\cat{E}\) be a displayed category over \(\cat{B}\) (not necessarily a fibration), and let \(f : X \to Y\) be a displayed isomorphism in \(\cat{E}\) above an isomorphism \(u : I \to J\) in \(\cat{B}\).
Suppose \(v : K \to I\) is a morphism in \(\cat{B}\) and \(g : Z \to Y\) is a morphism in \(\cat{E}\) above \(u \circ v : K \to J\). Then \(f^{-1} \circ g : Z \to X\) is the unique morphism in \(\cat{E}\) above \(v\) such that the following diagram commutes.
\begin{equation*}
\begin{tikzcd}[row sep=small]
Z \arrow[rd, dashed, "f^{-1} \circ g"] \arrow[rrd, bend left, "g"] \arrow[dd, -Triangle] \\
& X \arrow[r, "f"] & Y \arrow[dd, -Triangle] \\
K \arrow[rd, "v"] \arrow[rrd, bend left, near end, "u \circ v"] \\
& I \arrow[r, "u"] & J
\arrow[from=uul, to=l, -Triangle, crossing over]
\end{tikzcd}
\end{equation*}
Indeed, to verify uniqueness, suppose \(h : Z \to X\) is another morphism in \(\cat{E}\) above \(v\) such that \(f \circ h = g\), and observe that
\begin{equation*}
h = f^{-1} \circ f \circ h = f^{-1} \circ g.
\end{equation*}
Therefore \(f\) is Cartesian.
\end{partsolution}

\begin{remark}
The conclusion of \expartref{1}{1}{4}{ii} appears in the literature in, e.g., \cite[Lemma~8.1.4(i)]{MR1291599}
\end{remark}

\begin{partsolution}{ii}
Suppose \(\cat{E}\) is a displayed category over \(\cat{B}\) (not necessarily a fibration), \(u : I \to J\) and \(v : J \to K\) are morphisms in \(\cat{B}\), and \(f : X \to Y\) and \(g : Y \to Z\) are Cartesian morphisms in \(\cat{E}\) over \(u\) and \(v\), respectively.
To show that \(g \circ f : X \to Z\) is Cartesian over \(v \circ u : I \to K\), consider a morphism \(w : L \to I\) in \(\cat{B}\) and a morphism \(h : W \to Z\) in \(\cat{E}\) above \(v \circ u \circ w\).
These data are summarized in the following commutative diagram.
\begin{equation*}
\begin{tikzcd}[row sep=small]
W \arrow[rrrd, bend left, "h"] \arrow[dd, -Triangle] \\
& X \arrow[ddr, phantom," " {pullback}, very near start] \arrow[r, "f"] & Y \arrow[ddr, phantom," " {pullback}, very near start] \arrow[r, "g"] & Z \\
L \arrow[rd, "w"] \\
& I \arrow[r, "u"] & J \arrow[r, "v"] & K
\arrow[from=uull, to=ll, -Triangle, crossing over]
\arrow[from=uul, to=l, -Triangle, crossing over]
\arrow[from=uu, -Triangle, crossing over]
\end{tikzcd}
\end{equation*}
Since \(g\) is Cartesian, there exists a unique morphism \(h^\prime : W \to Y\) in \(\cat{E}\) above \(u \circ w\) such that \(h = g \circ h^\prime\), as in the following commutative diagram.
\begin{equation*}
\begin{tikzcd}[row sep=small]
W \arrow[rrrd, bend left, "h"] \arrow[rrd, dashed, bend left, near end, "h^\prime"] \arrow[dd, -Triangle] \\
& X \arrow[ddr, phantom," " {pullback}, very near start] \arrow[r, "f"] & Y \arrow[ddr, phantom," " {pullback}, very near start] \arrow[r, "g"] & Z \\
L \arrow[rd, "w"] \\
& I \arrow[r, "u"] & J \arrow[r, "v"] & K
\arrow[from=uull, to=ll, -Triangle, crossing over]
\arrow[from=uul, to=l, -Triangle, crossing over]
\arrow[from=uu, -Triangle, crossing over]
\end{tikzcd}
\end{equation*}
Moreover, since \(f\) is Cartesian, there exists a unique morphism \(h^{\prime\prime} : W \to X\) in \(\cat{E}\) above \(w\) such that \(h^\prime = f \circ h^{\prime\prime}\), as in the following commutative diagram.
\begin{equation*}
\begin{tikzcd}[row sep=small]
W \arrow[rrrd, bend left, "h"] \arrow[rrd, bend left, near end, "h^\prime"] \arrow[rd, dashed, "h^{\prime\prime}"] \arrow[dd, -Triangle] \\
& X \arrow[ddr, phantom," " {pullback}, very near start] \arrow[r, "f"] & Y \arrow[ddr, phantom," " {pullback}, very near start] \arrow[r, "g"] & Z \\
L \arrow[rd, "w"] \\
& I \arrow[r, "u"] & J \arrow[r, "v"] & K
\arrow[from=uull, to=ll, -Triangle, crossing over]
\arrow[from=uul, to=l, -Triangle, crossing over]
\arrow[from=uu, -Triangle, crossing over]
\end{tikzcd}
\end{equation*}
In particular, we have
\begin{equation*}
h
= g \circ h^\prime
= g \circ f \circ h^{\prime\prime}.
\end{equation*}
It remains to show that if \(i : W \to X\) is another morphism in \(\cat{E}\) above \(w\) such that \(h = g \circ f \circ i\), then \(i = h^{\prime\prime}\).
Indeed, in this case the uniqueness of \(h^\prime\) implies that \(f \circ i = h^\prime\), and so the uniqueness of \(h^{\prime\prime}\) implies that \(i = h^{\prime\prime}\).
Thus, \(g \circ f\) is Cartesian.
\end{partsolution}

\begin{partsolution}{iii}
Suppose \(\cat{E}\) is a displayed category over \(\cat{B}\) (not necessarily a fibration), \(u : I \to J\) and \(v : J \to K\) are morphisms in \(\cat{B}\), and \(f : X \to Y\) and \(g : Y \to Z\) are morphisms in \(\cat{E}\) over \(u\) and \(v\), respectively, such that \(g\) and \(g \circ f\) are Cartesian.
To show that \(f : X \to Y\) is Cartesian over \(u\), consider a morphism \(w : L \to I\) in \(\cat{B}\) and a morphism \(h : W \to Y\) in \(\cat{E}\) above \(u \circ w\).
This situation is summarized in the following commutative diagram.
\begin{equation*}
\begin{tikzcd}[row sep=small]
W \arrow[rrd, bend left, "h"] \arrow[dd, -Triangle] \\
& X \arrow[r, "f"] & Y \arrow[r, "g"] & Z \\
L \arrow[rd, "w"] \\
& I \arrow[r, "u"] & J \arrow[r, "v"] & K
\arrow[from=uull, to=ll, -Triangle, crossing over]
\arrow[from=uul, to=l, -Triangle, crossing over]
\arrow[from=uu, -Triangle, crossing over]
\end{tikzcd}
\end{equation*}
But then \(g \circ h : W \to Z\) is a morphism in \(\cat{E}\) above \(v \circ u \circ w\), so since \(g \circ f\) is Cartesian, there exists a unique morphism \(h^\prime : W \to X\) in \(\cat{E}\) over \(w\) such that \(g \circ h = g \circ f \circ h^\prime\).
Since \(g\) is Cartesian, \expartref{1}{1}{1}{i} implies that \(h = f \circ h^\prime\).
Moreover, suppose \(i : W \to X\) is another morphism above \(w\) such that \(h = f \circ i\).
Then \(g \circ h = g \circ f \circ i\), so the uniqueness of \(h^\prime\) arising from \(g \circ f\) being Cartesian implies that \(i = h^\prime\).
Thus, \(f\) is Cartesian.
\end{partsolution}

\begin{partsolution}{iv}
Let \(p : \cat{E} \to \cat{B}\) be a fibration; we will show that \(|p| : \Cart(\cat{E}) \to \cat{B}\) is also a fibration.
Consider a morphism \(u : I \to J\) in \(\cat{B}\) and an object \(Y\) in \(\cat{E}\) over \(J\).
Since \(p\) is a fibration, there exists a \(p\)-Cartesian morphism \(f : X \to Y\) in \(\cat{E}\) over \(u\).
We claim that \(f\) is \(|p|\)-Cartesian as well.
Indeed, suppose there is a morphism \(v : K \to I\) in \(\cat{B}\) and a \(p\)-Cartesian morphism \(h : Z \to Y\) in \(\cat{E}\) over \(u \circ v\).
Since \(f\) is \(p\)-Cartesian, there exists a unique morphism \(h : Z \to X\) in \(\cat{E}\) over \(v\) such that \(g = f \circ h\).
But now \partref{1}{1}{4}{iii} implies that \(h\) is \(p\)-Cartesian as well, so the entire diagram below is actually a diagram in the displayed category induced by \(|p| : \Cart(\cat{E}) \to \cat{B}\).
\begin{equation*}
\begin{tikzcd}[row sep=small]
Z \arrow[rd, dashed, "\exists! h"] \arrow[rrd, bend left, "g"] \arrow[dd, -Triangle] \\
& X \arrow[r, "f"] & Y \arrow[dd, -Triangle] \\
K \arrow[rd, "v"] \arrow[rrd, bend left, near end, "u \circ v"] \\
& I \arrow[r, "u"] & J
\arrow[from=uul, to=l, -Triangle, crossing over]
\end{tikzcd}
\end{equation*}
Therefore, \(f\) is \(|p|\)-Cartesian, so \(|p|\) is a fibration.
\end{partsolution}