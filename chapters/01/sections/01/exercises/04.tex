\begin{exercise}
Let
\begin{tikzcd}[row sep=small]
\cat{E} \arrow[d, "p"] \\ \cat{B}
\end{tikzcd}
be a fibration.
Prove that
\begin{parts}
\part
all isomorphisms in \(\cat{E}\) are Cartesian;
\part
if \(\xrightarrow{f}\xrightarrow{g}\) is a composable pair of Cartesian morphisms in \(\cat{E}\), then also their composite \(\xrightarrow{g \circ f}\) is Cartesian.
\end{parts}
Hence it makes sense to talk about the subcategory \(\Cart(\cat{E}) \hookrightarrow \cat{E}\) having all objects from \(\cat{E}\) but only the Cartesian arrows.
We write \(|p|\) for the composite \(\Cart(\cat{E}) \hookrightarrow \cat{E} \to \cat{B}\).
\begin{parts}[resume]
\part
Let \(\xrightarrow{f}\xrightarrow{g}\) be a composable pair in \(\cat{E}\) again.
Show now that if \(g\) and \(g \circ f\) are Cartesian, then \(f\) is Cartesian as well.
\part
Verify that a consequence of \partref{1}{1}{4}{iii} is that the functor \(|p| : \Cart(\cat{E}) \to \cat{B}\) is a fibration.
From \partref{1}{1}{3}{ii} in the previous exercise it follows that all fibres of \(|p|\) are groupoids (\emph{i.e.} that all maps in the fibres are isomorphisms).
\end{parts}
[This \(|p|\) will be called the \textbf{fibration of objects} of \(p\).]
\end{exercise}

We will prove this in the more general context of displayed categories (cf.~\cite{lmcs:5252}).

\begin{partsolution}{i}
Let \(\cat{E}\) be a fibration over \(\cat{B}\), and let \(f : X \to Y\) be a displayed isomorphism in \(\cat{E}\) above an isomorphism \(u : I \to J\) in \(\cat{B}\).
Suppose \(v : K \to I\) is a morphism in \(\cat{B}\) and \(g : Z \to Y\) is a morphism in \(\cat{E}\) above \(u \circ v : K \to J\). Then \(f^{-1} \circ g : Z \to X\) is the unique morphism in \(\cat{E}\) above \(v\) such that the following diagram commutes.
\begin{equation*}
\begin{tikzcd}[row sep=small]
Z \arrow[rd, dashed, "f^{-1} \circ g"] \arrow[rrd, bend left, "g"] \arrow[dd, -Triangle] \\
& X \arrow[r, "f"] & Y \arrow[dd, -Triangle] \\
K \arrow[rd, "v"] \arrow[rrd, bend left, near end, "u \circ v"] \\
& I \arrow[r, "u"] & J
\arrow[from=uul, to=l, -Triangle, crossing over]
\end{tikzcd}
\end{equation*}
Indeed, to verify uniqueness, suppose \(h : Z \to X\) is another morphism in \(\cat{E}\) above \(v\) such that \(f \circ h = g\), and observe that \(h = f^{-1} \circ f \circ h = f^{-1} \circ g\).
Therefore \(f\) is Cartesian.
\end{partsolution}