\begin{exercise}~ %
\begin{parts}
\part
Show that in the total category \(\simple(\cat{B})\) of a simple fibration
\begin{tikzcd}[row sep=small]
\simple(\cat{B}) \arrow[d] \\ \cat{B}
\end{tikzcd}
a morphism \((u, f) : (I, X) \to (J, Y)\) is Cartesian if and only if there is an isomorphism \(h : I \times X \overset{\cong}{\to} I \times Y\) in \(\cat{B}\) such that \(\pi \circ h = \pi\) and \(\pi^\prime \circ h = f\).
\part
Show that the assignment \((I, X) \mapsto I^*(X) = \begin{pmatrix}
\begin{tikzcd}[row sep=small]
I \times X \arrow[d, "\pi"] \\ I
\end{tikzcd}
\end{pmatrix}\)
extends to a full and faithful functor \(\simple(\cat{B}) \to \cat{B}^{\rightarrow}\).
Prove that it maps Cartesian morphisms to pullback squares.

[This functor restricts to a full and faithful functor \(\cat{B}\sslash I \to \cat{B} / I\).]
\end{parts}
\end{exercise}

\begin{remark}
In part~\partref{1}{3}{1}{i}, the assertion that ``there is an isomorphism \(h : I \times X \to I \times Y\) in \(\cat{B}\) such that \(\pi \circ h = \pi\) and \(\pi^\prime \circ h = f\)'' is the same as saying that ``\(\left\langle\pi, f\right\rangle : I \times X \to I \times Y\) is an isomorphism.''
\end{remark}

\begin{lemma}
\label{lemma:ex1.3.1}
Let \(\cat{B}\) be a category with objects \(A, B, C\), a product
\begin{equation*}
\begin{tikzcd}
A & A \times B \arrow[l, "\pi"{above}] \arrow[r, "\pi^\prime"{above}] & B
\end{tikzcd}
\end{equation*}
and morphisms \(f : C \to A\) and \(g : C \to B\) such that \(\langle f, g\rangle : C \to A \times B\) is an isomorphism.
Then
\begin{align*}
f \circ \left\langle f, g\right\rangle^{-1} &= \pi, &
g \circ \left\langle f, g\right\rangle^{-1} &= \pi^\prime.
\end{align*}
\end{lemma}

\begin{proof}
For the first case, we have
\begin{equation*}
f \circ \left\langle f, g\right\rangle^{-1}
= \pi \circ \left\langle f, g\right\rangle \circ \left\langle f, g\right\rangle^{-1}
= \pi.
\qedhere
\end{equation*}
\end{proof}

\begin{partsolution}{i}
(\(\Longrightarrow\))
Suppose \((u, f) : (I, X) \to (J, Y)\) is Cartesian in \(\simple(\cat{B})\) over \(u : I \to J\).
Consider the unique morphism \((\id, g) : (I, Y) \to (I, X)\) in \(\simple(\cat{B})\) over \(\id : I \to I\) such that the following diagram commutes.
\begin{equation*}
\begin{tikzcd}[row sep=small]
{(I, Y)} \arrow[dd, -Triangle] \arrow[drr, bend left=20, near end, "{(u, \pi^\prime)}"{above right}] \arrow[dr, dashed, "{(\id, g)}"{above right}] \\
& {(I, X)} \arrow[dd, -Triangle] \arrow[r, "{(u, f)}"{above}]
& {(J, Y)} \arrow[dd, -Triangle] \\
I \arrow[dr, "\id"{below left}] \\
& I \arrow[r, "u"]
& J
\end{tikzcd}
\end{equation*}
In particular,
\begin{equation*}
\pi^\prime = f \circ \left\langle \pi, g\right\rangle.
\end{equation*}
We will show that \(\left\langle\pi, f\right\rangle : I \times X \to I \times Y\) is an isomorphism in \(\cat{B}\) by showing that \(\left\langle\pi, g\right\rangle : I \times Y \to I \times X\) is its inverse.
On the one hand, we have
\begin{align*}
\left\langle\pi, f\right\rangle \circ \left\langle\pi, g\right\rangle
&= \left\langle \pi \circ \left\langle\pi, g\right\rangle, f \circ \left\langle\pi, g\right\rangle\right\rangle \\
&= \left\langle \pi, \pi^\prime\right\rangle \\
&= \id.
\end{align*}
On the other hand, we have
\begin{equation*}
f = \pi^\prime \circ \left\langle\pi, f\right\rangle,
\end{equation*}
so the upper triangle in the following diagram commutes, which makes the whole diagram commute.
\begin{equation*}
\begin{tikzcd}[row sep=small]
{(I, X)} \arrow[dd, -Triangle] \arrow[ddrrr, bend left, "{(u, f)}"{description}] \arrow[dr, "{\left(\id, f\right)}"{above right}] \\
& {(I, Y)}  \arrow[dd, -Triangle] \arrow[drr, bend left=20, "{(u, \pi^\prime)}"{description}] \arrow[dr, "{(\id, g)}"{above right}] \\
I \arrow[dr, "\id"{below left}]
& & {(I, X)} \arrow[dd, -Triangle] \arrow[r, "{(u, f)}"{above}]
& {(J, Y)} \arrow[dd, -Triangle] \\
& I \arrow[dr, "\id"{below left}] \\
& & I \arrow[r, "u"]
& J
\end{tikzcd}
\end{equation*}
Since \((u, f)\) is Cartesian, the morphism \((\id, g)\circ(\id, f)\) is the unique morphism \((I, X) \to (I, X)\) in \(\simple(\cat{B})\) over \(\id : I \to I\) such that
\begin{equation*}
(u, f) \circ (\id, g) \circ (\id, f)
= (u, f),
\end{equation*}
but \(\id = (\id, \pi^\prime) : (I, X) \to (I, X)\) is another such morphism, so
\begin{equation*}
(\id, \pi^\prime) = (\id, g)\circ(\id,f).
\end{equation*}
In particular,
\begin{equation*}
\pi^\prime = g \circ \left\langle\pi, f\right\rangle,
\end{equation*}
so
\begin{align*}
\left\langle\pi, g\right\rangle \circ \left\langle\pi, f\right\rangle
&= \left\langle \pi \circ \left\langle\pi, f\right\rangle, g \circ \left\langle\pi, f\right\rangle\right\rangle \\
&= \left\langle \pi, \pi^\prime\right\rangle \\
&= \id.
\end{align*}
Therefore, \(\left\langle\pi, f\right\rangle : I \times X \to I \times Y\) is an isomorphism in \(\cat{B}\).

(\(\Longleftarrow\))
Let \((u, f) : (I, X) \to (J, Y)\) be a morphism in \(\simple(\cat{B})\) over \(u : I \to J\), and suppose that \(\left\langle\pi, f\right\rangle : I \times X \to I \times Y\) is an isomorphism.
Consider another morphism \((v, g) : (K, Z) \to (J, Y)\) in \(\simple(\cat{B})\) and a morphism \(w : K \to I\) in \(\cat{B}\) such that \(u \circ w = v\).
This situation is summarized in the following commutative diagram.
\begin{equation*}
\begin{tikzcd}[row sep=small]
{(K, Z)} \arrow[dd, -Triangle] \arrow[drr, bend left=20, near end, "{(v, g)}"{above right}] \\
& {(I, X)} \arrow[dd, -Triangle] \arrow[r, "{(u, f)}"{above}]
& {(J, Y)} \arrow[dd, -Triangle] \\
K \arrow[dr, "w"{below left}] \arrow[drr, bend left=20, near end, "v"{above right}] \\
& I \arrow[r, "u"]
& J
\arrow[from=uul, to=l, -Triangle, crossing over]
\end{tikzcd}
\end{equation*}
Consider the following composition in \(\cat{B}\) (call it \(h : K \times Z \to X\)).
\begin{equation*}
\begin{tikzcd}
K \times Z \arrow[r, "{\left\langle w \circ \pi, g\right\rangle}"]
& I \times Y \arrow[r, "{\left\langle\pi, f\right\rangle^{-1}}"]
& I \times X \arrow[r, "{\pi^\prime}"]
& X
\end{tikzcd}
\end{equation*}
We claim that \((w, h) : (K, Z) \to (I, X)\) is the unique morphism in \(\simple(\cat{B})\) such that \((u, f) \circ (w, h) = (v, g)\), which will imply that \((u, f)\) is Cartesian.
That is, the following diagram will be shown to commute.
\begin{equation*}
\begin{tikzcd}[row sep=small]
{(K, Z)} \arrow[dd, -Triangle] \arrow[drr, bend left=20, near end, "{(v, g)}"{above right}] \arrow[dr, dashed, "{(w, h)}"{above right}] \\
& {(I, X)} \arrow[dd, -Triangle] \arrow[r, "{(u, f)}"{above}]
& {(J, Y)} \arrow[dd, -Triangle] \\
K \arrow[dr, "w"{below left}] \arrow[drr, bend left=20, near end, "v"{above right}] \\
& I \arrow[r, "u"]
& J
\arrow[from=uul, to=l, -Triangle, crossing over]
\end{tikzcd}
\end{equation*}
We have \(v = u \circ w\) by assumption, so to see that \((u, f) \circ (w, h) = (v, g)\), it remains to show that \(g = f \circ \langle w\circ\pi, h\rangle\).
Note that
\begin{align*}
\pi \circ \left\langle \pi, f\right\rangle^{-1} \circ \left\langle w \circ \pi, g\right\rangle
&= \pi \circ \left\langle w \circ \pi, g\right\rangle & &\text{by Lemma~\ref{lemma:ex1.3.1}} \\
&= w \circ \pi,
\end{align*}
so the universal property of products yields
\begin{equation}
\label{eq:ex1.3.1:tuple-identity}
\begin{aligned}
\left\langle \pi, f\right\rangle^{-1} \circ \left\langle w \circ \pi, g\right\rangle
&= \left\langle w \circ \pi, \pi^\prime \circ \left\langle\pi, f\right\rangle^{-1} \circ \left\langle w \circ \pi, g\right\rangle\right\rangle \\
&= \left\langle w \circ \pi, h\right\rangle.
\end{aligned}
\end{equation}
Then we have
\begin{align*}
g
&= \pi^\prime \circ \left\langle w \circ \pi, g\right\rangle \\
&= f \circ \left\langle \pi, f\right\rangle^{-1} \circ \left\langle w \circ \pi, g\right\rangle & &\text{by Lemma~\ref{lemma:ex1.3.1}} \\
&= f \circ \left\langle w \circ \pi, h\right\rangle. & &\text{by~\eqref{eq:ex1.3.1:tuple-identity}}
\end{align*}
To show uniqueness, suppose that \((w, h^\prime) : (K, Z) \to (I, X)\) is another morphism in \(\simple(\cat{B})\) over \(w: K \to I\) such that \((u, f) \circ (w, h^\prime) = (v, g)\) (and, in particular, \(g = f \circ \left\langle w \circ \pi, h^\prime\right\rangle\)).
Then we have
\begin{align*}
h^\prime
&= \pi^\prime \circ \left\langle w \circ \pi, h^\prime\right\rangle \\
&= \pi^\prime \circ \left\langle \pi, f\right\rangle^{-1} \circ \left\langle \pi, f\right\rangle \circ \left\langle w \circ \pi, h^\prime\right\rangle \\
&= \pi^\prime \circ \left\langle \pi, f\right\rangle^{-1} \circ \left\langle \pi \circ \left\langle w \circ \pi, h^\prime\right\rangle, f \circ \left\langle w \circ \pi, h^\prime\right\rangle\right\rangle \\
&= \pi^\prime \circ \left\langle \pi, f\right\rangle^{-1} \circ \left\langle  w \circ \pi, g\right\rangle \\
&= h.
\end{align*}
Therefore, \((u, f) : (I, X) \to (J, Y)\) is Cartesian in \(\simple(\cat{B})\) over \(u : I \to J\).
\end{partsolution}
