\begin{exercise}
The category \(\Sign\) captures signatures of functions.
A \textbf{signature of predicates} consists of a set of types \(T\) together with predicate symbols \(R : \sigma_1,\,\ldots,\,\sigma_n\) where each \(\sigma_i\) is a type (element of \(T\)).
Define an appropriate category of such signatures of predicates by change-of-base.
Define also a category with both function and predicate symbols by change-of-base.
[Such a category will be introduced in Definition~4.1.1.]
\end{exercise}

\begin{solution}
Recall that the category \(\Sign\) of multi-typed function signatures can be defined by the following change of base situation.
\begin{equation*}
\begin{tikzcd}
\Sign \arrow[r] \arrow[d] \arrow[rd, ""{pullback}, very near start, phantom]
& \Fam(\Sets) \arrow[d] \\
\Sets \arrow[r, "T \mapsto T^* \times T"] &
\Sets
\end{tikzcd}
\end{equation*}
In this case, the key for modeling function symbols is the functor \(T \mapsto T^* \times T\) which represents a template for constructing function signatures with a finite list of inputs (\(T^*\)) and one output (\(T\)) out of a set \(T\) of basic types.

The analogous functors generating predicate signatures and function-and-predicate signatures, repsectively, are
\begin{align*}
T & \mapsto T^* &
T & \mapsto T^* \times (T + 1) \cong (T^* \times T) + T^*.
\end{align*}
These yield change-of-base situations that define categories \(\Sign^\prime\) and \(\Sign^{\prime\prime}\) of multi-typed signatures of predicates and multi-typed signatures of functions and predicates, as follows.
\begin{equation*}
\begin{tikzcd}
{\Sign^\prime} \arrow[r] \arrow[d] \arrow[rd, ""{pullback}, very near start, phantom]
& \Fam(\Sets) \arrow[d] \\
\Sets \arrow[r, "T \mapsto T^*"] &
\Sets
\end{tikzcd}
\qquad
\begin{tikzcd}
{\Sign^{\prime\prime}} \arrow[r] \arrow[d] \arrow[rd, ""{pullback}, very near start, phantom]
& \Fam(\Sets) \arrow[d] \\
\Sets \arrow[r, "T \mapsto T^* \times (T + 1)"] &
\Sets
\end{tikzcd}
\qedhere
\end{equation*}
\end{solution}