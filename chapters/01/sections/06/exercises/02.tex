\begin{exercise}
A many-typed signature is called \textbf{finite} if it has only finitely many types and
function symbols.
Define the subcategory of \(\FinSign \hookrightarrow \Sign\) consisting of finite signatures by change-of-base.
\end{exercise}

\begin{solution}
Recall that \(\Sign\) can be defined via the change-of-base situation in the bottom right square of the following commutative diagram in \(\Cat\).
\begin{equation*}
\begin{tikzcd}
& & \FinFam(\Sets) \arrow[d, hookrightarrow] \\
& \Sign \arrow[dr, phantom, " "{pullback}, very near start] \arrow[r] \arrow[d]
& \Fam(\Sets) \arrow[d] \\
\FinSets \arrow[r, hookrightarrow]
& \Sets \arrow[r, "T \mapsto T^* \times T"{below}] & \Sets
\end{tikzcd}
\end{equation*}
This situation can be completed by pullbacks in \(\Cat\) as follows (noting that pullbacks of monic morphisms are monic; cf., e.g., \cite[Proposition 2.5.3]{MR1291599}).
\begin{equation*}
\begin{tikzcd}
\FinSign \arrow[dr, phantom, " "{pullback}, very near start] \arrow[d, hookrightarrow]  \arrow[r, hookrightarrow]
& \_ \arrow[dr, phantom, " "{pullback}, very near start] \arrow[d, hookrightarrow] \arrow[r]
& \FinFam(\Sets) \arrow[d, hookrightarrow] \\
\_ \arrow[dr, phantom, " "{pullback}, very near start] \arrow[d] \arrow[r, hookrightarrow]
& \Sign \arrow[dr, phantom, " "{pullback}, very near start] \arrow[r] \arrow[d]
& \Fam(\Sets) \arrow[d] \\
\FinSets \arrow[r, hookrightarrow]
& \Sets \arrow[r, "T \mapsto T^* \times T"{below}] & \Sets
\end{tikzcd}
\end{equation*}
This presentation recovers \(\FinSign\) as a subcategory of \(\Sign\), and the outer square is a pullback by \exref{1}{1}{5}, which simplifies to the following change-of-base situation.
\begin{equation*}
\begin{tikzcd}
\FinSign \arrow[dr, phantom, " "{pullback}, very near start] \arrow[r] \arrow[d]
& \FinFam(\Sets) \arrow[d] \\
\FinSets \arrow[r, "T \mapsto T^* \times T"{below}]
& \Sets
\end{tikzcd}
\qedhere
\end{equation*}
\end{solution}
