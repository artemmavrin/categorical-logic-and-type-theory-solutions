\begin{exercise}
Let \(\Sigma\) be a signature and \(T = |\Sigma|\) its set of types.
We write \(\SModel(\Sigma)\) for the fibre category over \(\Sigma\) of the fibration
\begin{tikzcd}[row sep=small]
\SModel \arrow[d] \\ \Sign
\end{tikzcd}.
This is the category of \(\Sigma\)-models.
\begin{parts}
\part
Show that the assignment
\begin{equation*}
X = \left(X_\sigma\right)_{\sigma \in T}
\mapsto \big(\Terms_\tau(X)\big)_{\tau \in T}
\end{equation*}
extends to a functor from the category \(\Fam\left(\Sets\right)_T = \Sets^T\) of \(T\)-indexed families of sets to \(\SModel(\Sigma)\).
\part
Assume \(\left(\left(A_\sigma\right)_{\sigma \in T},\left\llbracket\_\right\rrbracket\right)\) is a \(\Sigma\)-model.
Verify that an interpretation \(\left\llbracket\_\right\rrbracket_\rho^\tau : \Terms_\tau(X) \to A_\tau\) is a morphism of \(\Sigma\)-models.
\end{parts}
\end{exercise}

\begin{partsolution}{i}
We will define a ``term model'' functor
\begin{equation*}
\Terms : \Sets^T \to \SModel(\Sigma)
\end{equation*}
First, consider the following ``pre-functorial'' data (assignments of objects and morphisms in \(\Sets^T\) to objects and morphisms in \(\SModel(\Sigma)\)).
\begin{description}
\item[objects]
For a \(T\)-indexed family of sets \(X = \left(X_\sigma\right)_{\sigma \in T}\) (viewed as a set of variables for each of the basic types in \(T\)),
\begin{align*}
X \mapsto & \left(\left(\Terms_\tau(X)\right)_{\tau \in T}, \left\llbracket\_\right\rrbracket_X\right) \\
&= \text{the \textbf{term model} of \(\Sigma\) with variables from \(X\)}.
\end{align*}
That is, the right-hand side is the model of \(\Sigma\) whose \textbf{carrier sets} are \(\left(\Terms_\tau(X)\right)_{\tau \in T}\) and whose \textbf{interpretation function} \(\left\llbracket\_\right\rrbracket_X\) is given, for a function symbol \(F : \tau_0,\,\ldots,\,\tau_{n-1} \to \tau_n\) in \(\Sigma\), by
\begin{gather*}
\left\llbracket F \right\rrbracket_X : \Terms_{\tau_0}(X) \times \cdots \times \Terms_{\tau_{n-1}}(X) \to \Terms_{\tau_n}(X) \\
(M_0, \ldots, M_{n-1}) \mapsto F (M_0, \ldots, M_{n-1}).
\end{gather*}
\item[morphisms]
For a morphism \(f = \left(f_\sigma : X_\sigma \to Y_\sigma\right)_{\sigma \in T} : X \to Y\) in \(\Sets^T\),
\begin{equation*}
f \mapsto \left\llbracket\_\right\rrbracket_f : \left(\left(\Terms_\tau(X)\right)_{\tau \in T}, \left\llbracket\_\right\rrbracket_X\right) \to \left(\left(\Terms_\tau(Y)\right)_{\tau \in T}, \left\llbracket\_\right\rrbracket_Y\right)
\end{equation*}
defined as follows.
Viewing \(X\) as a set of variables used to interpret the term model \(\left(\left(\Terms_\tau(Y)\right)_{\tau \in T}, \left\llbracket\_\right\rrbracket_Y\right)\), define a \textbf{valuation}
\begin{align*}
\rho(f) &\coloneq \left(\rho_\sigma(f) : X_\sigma \to \Terms_\sigma(Y)\right)_{\sigma \in T} \\
\rho_\sigma(f) &\coloneq x \mapsto f_\sigma(x) \qquad\text{(as a term in \(\Terms_\sigma(Y)\))}.
\end{align*}
This yields an interpretation
\begin{align*}
\left\llbracket\_\right\rrbracket_f
\coloneq \left\llbracket\_\right\rrbracket_{\rho(f)}
= \left(\left\llbracket\_\right\rrbracket_{\rho(f)}^\tau : \Terms_\tau(X) \to \Terms_\tau(Y)\right)_{\tau \in T},
\end{align*}
which is a morphism of \(\Sigma\)-models from \(\left(\left(\Terms_\tau(X)\right)_{\tau \in T}, \left\llbracket\_\right\rrbracket_X\right)\) to \(\left(\left(\Terms_\tau(Y)\right)_{\tau \in T}, \left\llbracket\_\right\rrbracket_Y\right)\) by \partref{1}{6}{6}{ii}.
\end{description}
It remains to verify that these assignments are functorial.
First, induction on the structure of terms shows that
\begin{equation*}
\left\llbracket t\right\rrbracket_{\rho(\id_X)}^\tau
= t
\end{equation*}
for every term \(t \in \Terms_\tau(X)\) and every basic type \(\tau\), so that
\begin{equation*}
\left\llbracket\_\right\rrbracket_{\id_X}
= \id_{\left(\left(\Terms_\tau(X)\right)_{\tau \in T}, \left\llbracket\_\right\rrbracket_X\right)}.
\end{equation*}
Second, for morphisms \(f : X \to Y\) and \(g : Y \to Z\) in \(\Sets^T\), we have
\begin{align*}
\left\llbracket x \right\rrbracket_{\rho(g \circ f)}^\tau
&= \left(g \circ f\right)_\tau(x)
= g_\tau\left(f_\tau(x)\right)
= \left\llbracket \left\llbracket x \right\rrbracket_{\rho(f)}^\tau \right\rrbracket_{\rho(g)}^\tau
&\text{for \(x \in X_\tau\)}
\end{align*}
and, by induction on terms, for a function symbol \(F : \tau_0,\,\ldots,\,\tau_{n-1} \to \tau_n\) in \(\Sigma\) and terms \(M_0 \in \Terms_{\tau_0}(X),\ldots,M_{n-1} \in \Terms_{\tau_{n-1}}(X)\),
\begin{align*}
\left\llbracket F(M_0,\ldots,M_{n-1}) \right\rrbracket_{\rho(g \circ f)}^{\tau_n}
&= \left\llbracket F\right\rrbracket_Z\left(\left\llbracket M_0 \right\rrbracket_{\rho(g \circ f)}^{\tau_0},\ldots,\left\llbracket M_{n-1} \right\rrbracket_{\rho(g \circ f)}^{\tau_{n-1}}\right) \\
&= \left\llbracket F\right\rrbracket_Z\left(\left\llbracket \left\llbracket M_0 \right\rrbracket_{\rho(f)}^{\tau_0} \right\rrbracket_{\rho(g)}^{\tau_0},\ldots,\left\llbracket \left\llbracket M_{n-1} \right\rrbracket_{\rho(f)}^{\tau_{n-1}} \right\rrbracket_{\rho(g)}^{\tau_{n-1}}\right) \\
&= \left\llbracket F(\left\llbracket M_0 \right\rrbracket_{\rho(f)}^{\tau_0},\ldots,\left\llbracket M_{n-1}\right\rrbracket_{\rho(f)}^{\tau_{n-1}}) \right\rrbracket_{\rho(g)}^{\tau_n} \\
&= \left\llbracket \left\llbracket F\right\rrbracket_Y(\left\llbracket M_0 \right\rrbracket_{\rho(f)}^{\tau_0},\ldots,\left\llbracket M_{n-1}\right\rrbracket_{\rho(f)}^{\tau_{n-1}}) \right\rrbracket_{\rho(g)}^{\tau_n} \\
&= \left\llbracket \left\llbracket F(M_0,\ldots,M_{n-1})\right\rrbracket_{\rho(f)}^{\tau_n} \right\rrbracket_{\rho(g)}^{\tau_n}.
\end{align*}
We conclude that
\begin{equation*}
\left\llbracket\_\right\rrbracket_{g \circ f}
= \left\llbracket\_\right\rrbracket_g \circ \left\llbracket\_\right\rrbracket_f
\end{equation*}
so \(\Terms : \Sets^T \to \SModel(\Sigma)\) is a functor.
\end{partsolution}

\begin{partsolution}{ii}
Let \(\left(\left(A_\sigma\right)_{\sigma\in T}, \left\llbracket\_\right\rrbracket\right)\) be a \(\Sigma\)-model, \(X = \left(X_\sigma\right)_{\sigma \in T}\) a \(T\)-indexed family of sets (viewed as sets of variables for the basic types in \(T\)), and \(\rho = \left(\rho_\sigma : X_\sigma \to A_\sigma\right)_{\sigma \in T}\) a valuation.
Then \(\rho\) induces an \textbf{interpretation}
\begin{equation*}
\left\llbracket\_\right\rrbracket_\rho
\coloneq \left(\left\llbracket\_\right\rrbracket_\rho^\tau : \Terms_\tau(X) \to A_\tau\right)_{\tau \in T},
\end{equation*}
which is defined by induction on the structure of terms by
\begin{equation*}
\left\llbracket x \right\rrbracket_\rho^\tau \coloneq \rho_\tau(x)
\end{equation*}
for \(x \in X_\tau\) and
\begin{equation}
\label{eq:1.6.6:interpretation-definition}
\left\llbracket F(M_0,\ldots,M_{n-1}) \right\rrbracket_\rho^{\tau_n}
\coloneq \left\llbracket F \right\rrbracket \left(\left\llbracket M_0 \right\rrbracket_\rho^{\tau_0}, \ldots, \left\llbracket M_{n-1} \right\rrbracket_\rho^{\tau_{n-1}}\right)
\end{equation}
for function symbols \(F : \tau_0,\,\ldots,\,\tau_{n-1} \to \tau_n\) in \(\Sigma\) and terms \(M_i \in \Terms_{\tau_i}(X)\) (\(i \in \{0, \ldots, n - 1\}\)).
In fact, \eqref{eq:1.6.6:interpretation-definition} immediately yields
\begin{align*}
\left\llbracket F\right\rrbracket\left(\left\llbracket M_0\right\rrbracket_\rho^{\tau_0},\ldots,\left\llbracket M_{n-1}\right\rrbracket_\rho^{\tau_{n-1}}\right)
&= \left\llbracket F(M_0,\ldots,M_{n-1}) \right\rrbracket_\rho^{\tau_n} \\
&= \left\llbracket \left\llbracket F\right\rrbracket_X(M_0,\ldots,M_{n-1}) \right\rrbracket_\rho^{\tau_n}.
\end{align*}
In other words, the following diagram commutes for every function symbol \(F : \tau_0,\,\ldots,\,\tau_{n-1} \to \tau_n\) in \(\Sigma\), whence \(\left\llbracket\_\right\rrbracket_\rho\) is a morphism of \(\Sigma\)-models from the term model \(\left(\left(\Terms_\tau(X)\right)_{\tau \in T}, \left\llbracket\_\right\rrbracket_X\right)\) to \(\left(\left(A_\sigma\right)_{\sigma\in T}, \left\llbracket\_\right\rrbracket\right)\).
\begin{equation*}
\begin{tikzcd}
\Terms_{\tau_0}(X) \times\cdots\times \Terms_{\tau_{n-1}}(X)
\arrow[d, "\left\llbracket F\right\rrbracket_X"{left}]
\arrow[rr, "\left\llbracket\_\right\rrbracket_\rho^{\tau_0} \times\cdots\times \left\llbracket\_\right\rrbracket_\rho^{\tau_{n-1}}"]
&& A_{\tau_0} \times\cdots\times A_{\tau_{n-1}}
\arrow[d, "\left\llbracket F\right\rrbracket"{right}] \\
\Terms_{\tau_n}(X) \arrow[rr, "\left\llbracket\_\right\rrbracket_\rho^{\tau_n}"]
&& A_{\tau_n}
\end{tikzcd}
\qedhere
\end{equation*}
\end{partsolution}
