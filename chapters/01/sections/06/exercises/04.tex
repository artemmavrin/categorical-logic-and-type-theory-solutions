\begin{exercise}
Describe the category \(\SModel_{\ST}\) of models of single-typed signatures in detail.
\end{exercise}

\begin{solution}
Recall (\cite[Definition~1.6.7]{MR1674451}) that the category \(\SModel_{\ST}\) arises from the following stacked change-of-base situations.
\begin{equation*}
\begin{tikzcd}[column sep=small]
\SModel_{\ST} \arrow[drr, phantom, ""{pullback}, very near start] \arrow[rr, hookrightarrow] \arrow[d]
& & \SModel \arrow[d]
& \left(\Sigma, \left(A_\sigma\right)_{\sigma \in |\Sigma|}, \left\llbracket\_\right\rrbracket\right) \arrow[d, mapsto] \\
\Sign_{\ST} \arrow[drr, phantom, ""{pullback}, very near start] \arrow[rr, hookrightarrow] \arrow[d]
& & \Sign \arrow[d]
& \Sigma \arrow[d, mapsto] \\
1 \arrow[rr, "\_ \mapsto 1", hookrightarrow]
& & \Sets
& {|\Sigma|}
\end{tikzcd}
\end{equation*}
For the ``single-typed'' left side of this diagram, choose a basic type \(*\)\footnote{%
This is essentially the same as working in an ``untyped'' system; see \cite[Section~2.5]{MR1674451} or \cite[Section~21.4]{Harper2016}.}.

A single-typed signature \(\Sigma\) (in \(\Sign_{\ST}\)) is then just a pointwise-indexed family of sets \(\left(F_n\right)_{n \in \mathbb{N}}\), where each \(F_n\) is the set of function symbols with \(n\) inputs and one output (all having type \(*\)).
That is, for \(f \in F_n\) we have
\begin{equation*}
f : \underbrace{*,\,\ldots,\,*}_{\text{\(n\) times}} \to *.
\end{equation*}
A morphism from \(\Sigma = \left(F_n\right)_{n \in \mathbb{N}}\) to \(\Sigma^\prime = \left(F_n^\prime\right)_{n\in\mathbb{N}}\) in \(\Sign_{\ST}\) is an \(\mathbb{N}\)-indexed family of functions \(\left(\varphi_n : F_n \to F_n^\prime\right)_{n \in \mathbb{N}}\), so that a function symbol with \(n\) inputs in \(\Sigma\) yields a function symbol \(\varphi_n(f)\) with \(n\) inputs in \(\Sigma^\prime\).

Finally, the category \(\SModel_{\ST}\) of (set-theoretic) models of single-typed signatures consists of the following data.
\begin{description}
\item[objects]
Triples of the form \(\left(\Sigma, A, \left\llbracket\_\right\rrbracket\right)\), where \(\Sigma = \left(F_n\right)_{n \in \mathbb{N}}\) is a single-typed signature, \(A\) is the \textbf{carrier set} for the basic type \(*\), and for each \(n \in \mathbb{N}\) and function symbol \(f \in F_n\),
\begin{equation*}
\left\llbracket f \right\rrbracket : A^n \to A.
\end{equation*}
\item[morphisms]
A morphism
\begin{equation*}
\left(\varphi, H\right) : \left(\Sigma, A, \left\llbracket\_\right\rrbracket\right) \to \left(\Sigma^\prime, A^\prime, \left\llbracket\_\right\rrbracket^\prime\right)
\end{equation*}
in \(\SModel_{\ST}\) consists of a morphism of (single-typed) signatures \(\varphi : \Sigma \to \Sigma^\prime\) and a function \(H : A \to A^\prime\) such that the following diagram commutes for each \(n\)-ary function symbol \(f\) in \(\Sigma\).
\begin{equation*}
\begin{tikzcd}
A^n \arrow[r, "H^n"] \arrow[d, "\left\llbracket f\right\rrbracket"{left}]
& \left(A^\prime\right)^n \arrow[d, "\left\llbracket \varphi(f)\right\rrbracket^\prime"{right}] \\
A \arrow[r, "H"] & A^\prime
\end{tikzcd}
\end{equation*}
That is, for all \(a_0, \ldots, a_{n-1} \in A\) we have
\begin{equation*}
H\left(\left\llbracket f\right\rrbracket\left(a_0, \ldots, a_{n-1}\right)\right)
= \left\llbracket\varphi(f)\right\rrbracket^\prime\left(H(a_0), \ldots, H(a_{n-1})\right).
\qedhere
\end{equation*}
\end{description}
\end{solution}