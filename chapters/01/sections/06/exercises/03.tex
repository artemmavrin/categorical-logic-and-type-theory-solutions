\begin{exercise}
Many-typed signatures are sometimes defined (like in [343] or in [282, 2.2.1]) as objects of the category \(\Sign^\prime\) which arises in the following change-of-base situation.
\begin{equation*}
\begin{tikzcd}
\Sign^\prime \arrow[dr, phantom, " "{pullback}, very near start] \arrow[r] \arrow[d]
& \Sets^{\rightarrow} \arrow[d, "\cod"{right}] \\
\Sets \arrow[r, "T \mapsto T^* \times T"{below}]
& \Sets
\end{tikzcd}
\end{equation*}
\begin{parts}
\part
Describe the category \(\Sign^\prime\) in elementary terms.
\part
Show that the categories \(\Sign\) and \(\Sign^\prime\) are equivalent.
\item
One often prefers \(\Sign\) to \(\Sign^\prime\) because signatures in \(\Sign\) allow \textbf{overloading} of function symbols: for example the use of \(+\) both for addition
of integers and for addition of reals.
Explain.
\end{parts}
[Another advantage of \(\Sign\) is that
\begin{tikzcd}[row sep=small]
\Sign \arrow[d] \\ \Sets
\end{tikzcd}
is a \emph{split} fibration.]
\end{exercise}

\begin{partsolution}{i}
Objects in \(\Sign^\prime\) are ``display-indexed'' families
\begin{equation*}
\mathcal{F} : X \to T^* \times T,
\end{equation*}
where \(T\) is an ordinary set of \textbf{basic types} and \(X\) is the total space of \textbf{function symbols}.
The fibre \(\mathcal{F}^{-1}\left(\left\langle\left\langle\sigma_0,\ldots,\sigma_{n-1}\right\rangle,\sigma_n\right\rangle\right) \subseteq X\) over an index \(\left\langle\left\langle\sigma_0,\ldots,\sigma_{n-1}\right\rangle,\sigma_n\right\rangle \in T^* \times T\) is the set of function symbols with \(n\) inputs of types \(\sigma_0, \ldots, \sigma_{n-1}\), respectively, and an output of type \(\sigma_n\).

A morphism in \(\Sign^\prime\) between \(\mathcal{F} : X \to T^* \times T\) and \(\mathcal{G} : Y \to S^* \times S\) is a pair \((u, f)\) where \(u : T \to S\) and \(f : X \to Y\) are functions such that the following diagram commutes
\begin{equation*}
\begin{tikzcd}
X \arrow[r, "f"] \arrow[d, "\mathcal{F}"{left}]
& Y \arrow[d, "\mathcal{G}"{right}] \\
T^* \times T \arrow[r, "u^* \times u"]
& S^* \times S
\end{tikzcd}
\qedhere
\end{equation*}


\end{partsolution}