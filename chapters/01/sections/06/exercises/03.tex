\begin{exercise}
Many-typed signatures are sometimes defined (like in [343] or in [282, 2.2.1]) as objects of the category \(\Sign^\prime\) which arises in the following change-of-base situation.
\begin{equation*}
\begin{tikzcd}
\Sign^\prime \arrow[dr, phantom, " "{pullback}, very near start] \arrow[r] \arrow[d]
& \Sets^{\rightarrow} \arrow[d, "\cod"{right}] \\
\Sets \arrow[r, "T \mapsto T^* \times T"{below}]
& \Sets
\end{tikzcd}
\qedhere
\end{equation*}
\begin{parts}
\part
Describe the category \(\Sign^\prime\) in elementary terms.
\part
Show that the categories \(\Sign\) and \(\Sign^\prime\) are equivalent.
\item
One often prefers \(\Sign\) to \(\Sign^\prime\) because signatures in \(\Sign\) allow \textbf{overloading} of function symbols: for example the use of \(+\) both for addition
of integers and for addition of reals.
Explain.
\end{parts}
[Another advantage of \(\Sign\) is that
\begin{tikzcd}[row sep=small]
\Sign \arrow[d] \\ \Sets
\end{tikzcd}
is a \emph{split} fibration.]
\end{exercise}