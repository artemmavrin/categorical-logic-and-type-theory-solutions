\begin{exercise}
Prove that the following `extended' structural rules are derivable.
\begin{parts}
\part
\makebox[\linewidth]{%
\AxiomC{\(\Gamma, v_n : \sigma, \Delta, v_{n+m} : \rho, \Theta \vdash M : \tau\)}
\UnaryInfC{\(\Gamma, v_n : \rho, \Delta, v_{n+m} : \sigma, \Theta \vdash M\left[v_n/v_{n+m},v_{n+m}/v_n\right] : \tau\)}
\DisplayProof}
\part
\makebox[\linewidth]{%
\AxiomC{\(\Gamma, v_n : \sigma, \Delta \vdash M : \tau\)}
\UnaryInfC{\(\Gamma, \Delta, v_{n+m} : \sigma \vdash M\left[v_{n+m}/v_n, v_n/v_{n+1},\ldots,v_{n+m-1}/v_{n+m}\right] : \tau\)}
\DisplayProof}
\part
\makebox[\linewidth]{%
\AxiomC{\(\Gamma, v_n : \sigma, \Delta, v_{n+m} : \sigma, \Theta \vdash M : \tau\)}
\UnaryInfC{%
\begin{tikzpicture}
\node(1){\(\Gamma, v_n : \sigma, \Delta, \Theta \vdash\)};
\node[below right = -1mm and -23mm of 1]{\(M\left[v_n/v_{n+m},v_{n+m}/v_{n+m+1},\ldots,v_{n+m+k-1}/v_{n+m+k}\right] : \tau\)};
\end{tikzpicture}}
\DisplayProof}
\part
\makebox[\linewidth]{%
\AxiomC{\(\Gamma, \Gamma \vdash M : \tau\)}
\UnaryInfC{\(\Gamma \vdash M[v_1/v_{n+1}, \ldots, v_n / v_{2n}] : \tau\)}
\DisplayProof}
\end{parts}
\end{exercise}

\begin{lemma}
Let \(\sigma\) and \(\rho\) be basic types and \(\Gamma, \Delta, \Theta\) contexts of lengths \(m, n, p\), respectively.
Then the following two rules are derivable\footnote{Recall that, by convention in \cite{MR1674451}, context variable indices start at \(1\).}.
\begin{enumerate}
\item
\label{lemma:ex:2.1.1:1.a}
\makebox[\linewidth]{%
\AxiomC{\(\Gamma, v_{m+1} : \sigma, \Delta, \Theta \vdash M : \tau\)}
\UnaryInfC{\(\Gamma, \Delta, v_{m+n+1} : \sigma, \Theta \vdash M\left[\sfrac{v_{m+1}}{v_{m+n+1}},\sfrac{v_{m+2}}{v_{m+1}},\ldots,\sfrac{v_{m+n+1}}{v_{m+n}}\right] : \tau\)}
\DisplayProof}
\item
\label{lemma:ex:2.1.1:1.b}
\makebox[\linewidth]{%
\AxiomC{\(\Gamma, \Delta, v_{m+n+1} : \sigma, \Theta \vdash M : \tau\)}
\UnaryInfC{\(\Gamma, v_{m+1} : \sigma, \Delta, \Theta \vdash M\left[\sfrac{v_{m+1}}{v_{m+2}},\ldots,\sfrac{v_{m+n}}{v_{m+n+1}},\sfrac{v_{m+n+1}}{v_{m+1}}\right] : \tau\)}
\DisplayProof}
\end{enumerate}
\end{lemma}

We will prove \ref{lemma:ex:2.1.1:1.a} by applying the \textbf{exchange} rule \(n\) times to shift the variable of type \(\sigma\) (initially in position \(m+1\)) from left to right one context position at a time.
We omit the proof of \ref{lemma:ex:2.1.1:1.b} since it involves the same argument but in the right-to-left direction.

\begin{proof}[Proof of \ref{lemma:ex:2.1.1:1.a}]
We proceed by induction on \(n\), the length of \(\Delta\).
If \(n = 0\), then the conclusion is vacuously true.
If \(n > 0\), then, writing \(\Delta = \Delta^\prime, v_n : \sigma^\prime\), we first apply \ref{lemma:ex:2.1.1:1.a} inductively to shift the variable of type \(\sigma\) to the right past \(\Delta^\prime\) and then apply the \textbf{exchange}/) rule to shift it right one more time:
\begin{prooftree}
\AxiomC{\(\Gamma, v_{m+1} : \sigma, \Delta, \Theta \vdash M : \tau\)}
\RightLabel{\ref{lemma:ex:2.1.1:1.a}}
\UnaryInfC{%
\begin{tikzpicture}
\node(1){\(\Gamma, \Delta^\prime, v_{m+n} : \sigma, v_{m+n+1} : \sigma^\prime, \Theta \vdash\)};
\node[below right = -1mm and -33mm of 1]{\(M\left[\sfrac{v_{m+1}}{v_{m+n}},\sfrac{v_{m+2}}{v_{m+1}},\ldots,\sfrac{v_{m+n}}{v_{m+n-1}}\right] : \tau\)};
\end{tikzpicture}}
\RightLabel{(E)}
\UnaryInfC{\(\Gamma, \Delta, v_{m+n+1} : \sigma, \Theta \vdash M\left[\sfrac{v_{m+1}}{v_{m+n+1}},\sfrac{v_{m+2}}{v_{m+1}},\ldots,\sfrac{v_{m+n+1}}{v_{m+n}}\right] : \tau\)}
\end{prooftree}
The last step required computing the consecutive context permutation
\begin{align*}
& M\left[\sfrac{v_{m+1}}{v_{m+n}},\sfrac{v_{m+2}}{v_{m+1}},\ldots,\sfrac{v_{m+n}}{v_{m+n-1}}\right]\left[\sfrac{v_{m+n}}{v_{m+n+1}},\sfrac{v_{m+n+1}}{v_{m+n}}\right] \\
& \qquad = M\left[\sfrac{v_{m+1}}{v_{m+n+1}},\sfrac{v_{m+2}}{v_{m+1}},\ldots,\sfrac{v_{m+n+1}}{v_{m+n}}\right].
\qedhere
\end{align*}
\end{proof}
