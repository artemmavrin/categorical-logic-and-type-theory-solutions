\begin{exercise}
Prove that the following `extended' structural rules are derivable.
\begin{parts}
\part
\makebox[\linewidth]{%
\AxiomC{\(\Gamma, v_n : \sigma, \Delta, v_{n+m} : \rho, \Theta \vdash M : \tau\)}
\UnaryInfC{\(\Gamma, v_n : \rho, \Delta, v_{n+m} : \sigma, \Theta \vdash M\left[v_n/v_{n+m},v_{n+m}/v_n\right] : \tau\)}
\DisplayProof}
\part
\makebox[\linewidth]{%
\AxiomC{\(\Gamma, v_n : \sigma, \Delta \vdash M : \tau\)}
\UnaryInfC{\(\Gamma, \Delta, v_{n+m} : \sigma \vdash M\left[v_{n+m}/v_n, v_n/v_{n+1},\ldots,v_{n+m-1}/v_{n+m}\right] : \tau\)}
\DisplayProof}
\part
\makebox[\linewidth]{%
\AxiomC{\(\Gamma, v_n : \sigma, \Delta, v_{n+m} : \sigma, \Theta \vdash M : \tau\)}
\UnaryInfC{%
\begin{tikzpicture}
\node(1){\(\Gamma, v_n : \sigma, \Delta, \Theta \vdash\)};
\node[below right = -1mm and -23mm of 1]{\(M\left[v_n/v_{n+m},v_{n+m}/v_{n+m+1},\ldots,v_{n+m+k-1}/v_{n+m+k}\right] : \tau\)};
\end{tikzpicture}}
\DisplayProof}
\part
\makebox[\linewidth]{%
\AxiomC{\(\Gamma, \Gamma \vdash M : \tau\)}
\UnaryInfC{\(\Gamma \vdash M[v_1/v_{n+1}, \ldots, v_n / v_{2n}] : \tau\)}
\DisplayProof}
\end{parts}
\end{exercise}

\begin{lemma}[Permutation Rule]
\label{lemma:ex:2.1.1:permutation}
If \(\Gamma = v_1 : \sigma_1, \ldots, v_n : \sigma_n\) and \(\pi\) is a permutation of \(\{1,\ldots,n\}\), then the following structural rule is derivable:
\begin{prooftree}
\Axiom$\Gamma \ \fCenter\vdash M : \tau$
\RightLabel{(P)}
\UnaryInf$\pi(\Gamma) \ \fCenter\vdash M\left[\pi\right] : \tau$
\end{prooftree}
where
\begin{align*}
\pi(\Gamma) &\coloneq v_1 : \sigma_{\pi(1)}, \ldots, v_n : \sigma_{\pi(n)}, &
M\left[\pi\right] &\coloneq M\left[\sfrac{v_1}{v_{\pi(1)}}, \ldots, \sfrac{v_n}{v_{\pi(n)}}\right].
\end{align*}
\end{lemma}

\begin{proof}
Every permutation of finitely many items can be represented as a composition of zero or more adjacent transpositions\footnote{E.g., see \cite[Chapter~I, \S5, No.~7, Proposition~9]{BourbakiAlgebraI}.}, so we proceed by induction on the number of adjacent transpositions in such a representation.
First, the composition of zero adjacent transpositions is the identity permutation, for which the conclusion holds vacuously.
Next, if \(\pi = [i \mapsto i + 1, i + 1\mapsto i] \circ \pi^\prime\) for some \(i \in \{1, \ldots, n - 1\}\), then we deduce by induction
\begin{prooftree}
\AxiomC{\(v_1 : \sigma_1, \ldots, v_n : \sigma_n \vdash M : \tau\)}
\RightLabel{(P)}
\UnaryInfC{\(v_1 : \sigma_{\pi^\prime(1)}, \ldots, v_i : \sigma_{\pi^\prime(i)}, v_{i+1} : \sigma_{\pi^\prime(i+1)}, \ldots, v_n : \sigma_{\pi^\prime(n)} \vdash M\left[\pi^\prime\right] : \tau\)}
\RightLabel{(E)}
\UnaryInfC{\footnotesize\(v_1 : \sigma_{\pi^\prime(1)}, \ldots, v_i : \sigma_{\pi^\prime(i+1)}, v_{i+1} : \sigma_{\pi^\prime(i)}, \ldots, v_n : \sigma_{\pi^\prime(n)} \vdash M\left[\pi^\prime\right]\left[\sfrac{v_i}{v_{i+1}},\sfrac{v_{i+1}}{v_i}\right] : \tau\)}
\end{prooftree}
where, since \(\pi = [i \mapsto i + 1, i + 1\mapsto i] \circ \pi^\prime\), we have
\begin{align*}
&v_1 : \sigma_{\pi^\prime(1)}, \ldots, v_i : \sigma_{\pi^\prime(i+1)}, v_{i+1} : \sigma_{\pi^\prime(i)}, \ldots, v_n : \sigma_{\pi^\prime(n)} \\
&\qquad\qquad\qquad\qquad= v_1 : \sigma_{\pi(1)}, \ldots, v_n : \sigma_{\pi(n)},
\end{align*}
and
\begin{equation*}
M\left[\pi^\prime\right]\left[\sfrac{v_i}{v_{i+1}},\sfrac{v_{i+1}}{v_i}\right]
= M\left[\pi\right]
\qedhere
\end{equation*}
\end{proof}

\begin{partsolution}{i}
Apply the permutation rule (Lemma~\ref{lemma:ex:2.1.1:permutation}) with the permutation that swaps \(n\) and \(n+m\).
\end{partsolution}

\begin{partsolution}{ii}
Apply the permutation rule (Lemma~\ref{lemma:ex:2.1.1:permutation}) with the cyclic permutation
given by \(n \mapsto n + 1, \ldots, n + m - 1 \mapsto n + m, n + m \mapsto n\).
\end{partsolution}
